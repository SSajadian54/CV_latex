\documentclass[11pt,a4paper,sans]{moderncv} % Font sizes: 10, 11, or 12; paper sizes: a4paper, letterpaper, a5paper, legalpaper, executivepaper
\usepackage{standalone}
\moderncvstyle{classic} % CV theme - options include: 'casual' (default), 'classic', 'oldstyle' and 'banking'
\moderncvcolor{blue} % CV color - options include: 'blue' (default), 'orange', 'green', 'red', 'purple', 'grey' and 'black'
\usepackage{lipsum} % Used for inserting dummy 'Lorem ipsum' text into the template
\usepackage[scale=0.85]{geometry} % Reduce document margins
\usepackage{enumitem}  
\renewcommand\refname{}
\usepackage{etoolbox}
\usepackage[dvipsnames]{xcolor}
\PassOptionsToPackage{x11names}{xcolor}
\definecolor{blue1}{rgb}{0.0, 0.53, 0.74}
\definecolor{blue2}{rgb}{0.0, 0.18, 0.39}


\vspace*{-3em} % squish the references together
\usepackage[style=authoryear,sorting=ydnt,dashed=true]{biblatex}%backend=biber,style=numeric]
\renewbibmacro*{date}{}
\renewbibmacro*{date+extrayear}{}
\renewbibmacro*{issue+date}{}
\newcommand*{\bibyear}{}
\defbibenvironment{bibliography}
{\list
 {\iffieldequals{year}{\bibyear}
		{}
		{\printfield{year}%
			\savefield{year}{\bibyear}} }
	{\setlength{\topsep}{0pt}% layout parameters based on moderncvstyleclassic.sty
		\setlength{\labelwidth}{\hintscolumnwidth}%
		\setlength{\labelsep}{\separatorcolumnwidth}%
		\setlength{\itemsep}{\bibitemsep}%
		\leftmargin\labelwidth%
		\advance\leftmargin\labelsep}%
	\sloppy\clubpenalty4000\widowpenalty4000}
{\endlist}
{\item}

\addbibresource{journal.bib}
\newcommand{\bibsection}{}
\newcommand{\mnras}{\textbf{\textcolor{blue2}{MNRAS}}}
\newcommand{\aj}{\textbf{\textcolor{blue2}{AJ}}}
\newcommand{\apj}{\textbf{\textcolor{blue2}{ApJ}} }
\newcommand{\aap}{\textbf{\textcolor{blue2}{A \&A}} }
\newcommand{\ijpr}{\textbf{\textcolor{blue2}{IJPR}} }

\firstname{\textbf{Sedighe}\\}% Your first name
\familyname{\textbf{Sajadian}} % Your last name
\title{Curriculum Vitae}
\address{Department of Physics}{Isfahan University of Technology}
\mobile{(+98) 9179965712}
\email{s.sajadian@iut.ac.ir} 
\homepage{https://sajadian.iut.ac.ir/}{https://sajadian.iut.ac.ir/}
%\extrainfo{\faGithub\href{https://github.com/SSajadian54}{https://github.com/SSajadian54}}
%\faGithub\href{https://github.com/}{SSajadian54} 
%\extrainfo{\faGithub\href{https://github.com/}{SSajadian54}}% <===============
\extrainfo{%
\httplink{www.linkedin.com/in/sedighe-sajadian-9020a237/}\\
\httplink{github.com/SSajadian54}
}
%\extrainfo{\faGithub\href{https://github.com/}{ name}}% <===============
%\extrainfo{\faLinkedin\href{https://www.linkedin.com/in/sedighe-sajadian-9020a237/}{linkedin.com/in/sedighe-sajadian-9020a237}}
\photo[85pt][0.35pt]{picture}

\newcommand{\cvdoublecolumn}[2]{%
  \cvitem[.75em]{}{
    \begin{minipage}[t]{\listdoubleitemcolumnwidth}#1\end{minipage}%
    \hfill\usepackage{xcolor}
    \begin{minipage}[t]{\listdoubleitemcolumnwidth}#2\end{minipage}} }
\newcommand\Colorhref[3][orange]{\href{#2}{\small\color{#1}#3}}
%----------------------------------------------------------------------------------------


\begin{document}
\makecvtitle 
%\lipsum

%%%%%%%%%%%%%%%%%%%%%%%%%%%%%%%%%%%%%%%%%
\section{Education}
\cventry{2007--2011: }{\textcolor{blue2}{PhD of Physics, Astrophysics}}{Physics Department, Sharif University of Technology}{Tehran, Iran}{}
{Studying detection of exo-planet by microlensing; Ph.D. Supervisor: S. Rahvar} \cvitem{CGPA: }{18.37/20}
\cventry{2005--2007: }{\textcolor{blue2}{Master of Physics, Particle physics}}{Physics department, Sharif University of Technology}{Tehran, Iran}{}
{Studying dynamic symmetry of Hydrogen atom; M.Sc. Supervisor: F. Ardalan} \cvitem{CGPA: }{18.09/20}
\cventry{2001--2005: }{\textcolor{blue2}{Bachelor of Physics}}{Shahid-Chamran University}{Ahvaz, Iran}{}{} \cvitem{CGPA: }{18.27/20}

%%%%%%%%%%%%%%%%%%%%%%%%%%%%%%%%%%%%%%%%%
\begin{section}{Employment History}
%\cvitem{2022-present}{\textcolor{blue2}{\textbf{Student Vice Chair}}, Physics department, Isfahan University of Technology, Isfahan, Iran.}		
\cvitem{2017-present}{\textcolor{blue2}{\textbf{Assistant professor}}, Physics department, Isfahan University of Technology, Isfahan, Iran.}	
\cvitem{2015-2016}{\textcolor{blue2}{\textbf{Postdoctoral researcher}}, Physics department, Sharif University of Technology, Tehran, Iran.}	
\cvitem{2011-2014}{\textcolor{blue2}{\textbf{Postdoctoral researcher}}, School of Astronomy, IPM, Tehran, Iran.}	
\end{section} 
 
%%%%%%%%%%%%%%%%%%%%%%%%%%%%%%%%%%%%%%%%%
\section{Professional Memberships}
\cvitem{2022, 22 Dec (comming)}{Chair of organizing committee in \textcolor{blue2}{\textbf{\textit{Physics Day} Workshop}} for high school students from Isfahan province, Physics Department, Isfahan University of Technology(Outreach activity)}
\cvitem{2022, 28-29 July}{Chair of scientific committee in \textcolor{blue2}{\textbf{\textit{Transient events and Multi-messenger astrophysics} Workshop}}, Physics Department, Isfahan University of Technology in cooperation with INO, and IPM.}
\cvitem{2018, Jan}{Chair of organizing committee in \textcolor{blue2}{\textbf{\textit{National conference on gravitation and cosmology}}}, Physics Department, Isfahan University of Technology.}
\cvitem{2016, Dec}{Organizer and lecturer in \textcolor{blue2}{\textbf{\textit{Searching stellar brightness curves}~Workshop}} in Physics Department, Sharif University of Technology, Tehran.}
\cvitem{2016 --present}{\textcolor{blue2}{\textbf{Member of MiNDSTEp consortium}}, a follow-up observing group for on-going microlensing events with the Danish 1.54 m telescope.}
\cvitem{2017 --present}{Reviewer of \textcolor{blue2}{\textbf{MNRAS, MNRAS Letter, The Astronomical Journal, and Iranian Journal of Physics Research}}. }

%%%%%%%%%%%%%%%%%%%%%%%%%%%%%%%%%%%%%%%%%
\begin{section}{Professional Experience}
%\cvitem{}{\textit{\textbf{}} .}	
\cvitem{2017-present }{\textcolor{blue2}{\textbf{Data-reduction}} of images taken by the Lucky Imaging camera on the Danish telescope with DANDIA pipeline.}	
\cvitem{2021}{Working with \textcolor{blue2}{\textbf{IRAF/ DAOPHOT}} for images taken with the Danish telescope.}
\cvitem{2022}{Using \textcolor{blue2}{\textbf{Supervised Machine Learning}} for microlensing data analysis and for Gaia data.}
\cvitem{2016-present}{\textcolor{blue2}{\textbf{On-site and remote observations}}, Danish 1.54 m telescope, La Silla observatory, Chile.}	
\end{section} 
\clearpage


%%%%%%%%%%%%%%%%%%%%%%%%%%%%%%%%%%%%%%%%%
\begin{section}{Fellowships, Awards, \&  Honors} 
\subsection{Grants}	
\cvitem{2020-2021}{Research \textcolor{blue2}{\textbf{Grant}} from Prof.~Han, Department of Physics, Chungbuk National univbersity. I was his researcher for one year.}
\cvitem{2017-2019}{\textcolor{blue2}{\textbf{Grant}} for accomplishing a research project \textcolor{blue2}{Extra solar planets: detection, formation}, from \textit{Iran Science National Foundation(ISNF)}, No: INSF-95843339.}	
\cvitem{2016-2017}{\textcolor{blue2}{\textbf{Grant}} for accomplishing a research project \textcolor{blue2}{polarimetry microlensing}, from \textit{Iran Science National Foundation (ISNF)}, No: INSF-94017434.}	
\cvitem{2015-2016}{\textcolor{blue2}{\textbf{Grant}} for publishing ISI papers in Q1 journals, from \textit{Iran Science Elites Federation}.}	
\subsection{Honors}
\cvitem{2005}{Ranked 54th in the \textcolor{blue2}{\textbf{M.Sc. Qualifying Exam}} with more 50000 participants. }
\cvitem{2001-2005}{Ranked 1th during \textcolor{blue2}{\textbf{B.Sc. program in Physics Department}}, Shahid-Chamran University.}
\cvitem{2006--2010}{Membership in \textcolor{blue2}{\textbf{Exceptional Talent Academy}}, Sharif University of Technology, Tehran.}
\cvitem{2003--2005}{Membership in \textcolor{blue2}{\textbf{Exceptional Talent Academy}}, Shahid-Chamran University, Ahvaz.}
\end{section} 
 

%%%%%%%%%%%%%%%%%%%%%%%%%%%%%%%%%%%%%%%%% 
\begin{section}{Teaching} 
\cvitem{2022, Spring}{\textcolor{blue2}{\textbf{Analytical mechanic I}} for undergraduate students, physics department, Isfahan University of Technology.}	
\cvitem{2018, 2019, 2022}{\textcolor{blue2}{\textbf{Astrophysics}} for undergraduate students, Physics Department, Isfahan University of Technology.}	
\cvitem{2019, 2021, 2022}{\textcolor{blue2}{\textbf{Cosmology \& Special topics in cosmology}} for graduate students, Physics Department, Isfahan University of Technology.}
\cvitem{2017-present}{\textcolor{blue2}{\textbf{Fundamental Physics I (calculus-based)}} fall semesters for undergraduate and engineering students, Isfahan University of Technology.}	
\cvitem{2017-present}{\textcolor{blue2}{\textbf{Fundamental Physics II (calculus-based)}} spring semesters for undergraduate and engineering students, Isfahan University of Technology.}	
\end{section} 

%%%%%%%%%%%%%%%%%%%%%%%%%%%%%%%%%%%%%%%%% 
\begin{section}{Supervising projects}
	\subsection{\textbf{Bachelor projects}}
	\cvitem{2022-}{\textit{\textbf{Setareh Moein}}; Thesis: Studying stellar atmosphere modelling with \textit{MARCS} code. } 
	\cvitem{2022, Sep}{\textit{\textbf{Hossein Fatheddin}}; Thesis: New method to solve binary-lens equation (\textbf{Results published}).}
	\cvitem{2021, Nov}{\textit{\textbf{Ehsan Goreishi}}; Thesis: Studying age-velocity relation in Gaia data.}
	\cvitem{2020, Sep}{\textit{\textbf{Melika Sarrami}}; Thesis: Finite-source effect in shoft-duration microlensing events.}
	\cvitem{2020, Sep}{\textit{\textbf{Mahshad Rashidi}}; Thesis: Characterizing microlensing light curves from spotted stars (\textbf{Results published}).}
	\cvitem{2019, July}{\textit{\textbf{Ali Salehi}}; Thesis: Detecting inner regions of protoplanetary discs around sources of microlensing with WFIRST Survey (\textbf{Results published}).}
	\subsection{\textbf{Master projects}}
	\cvitem{2022-}{\textit{\textbf{Sina Hematian}}; Thesis: Applications of Machine Learning approaches in microlensing observations}
	 \cvitem{2022-}{\textit{\textbf{Aref Asadi}}; Thesis: Studying consistency between Gaia data and Besancon model}
    \cvitem{2021-}{\textit{\textbf{AliReza  Zareshahi}}; Thesis: On the detection of free-floating moon-planet systems through microlensing observations.}
	\cvitem{2021-}{\textit{\textbf{AmirAli Tavajoh}}; Thesis: Improving Newton's and Laguerre's methods for solving binary-lens equations.}
	\cvitem{2021}{\textit{\textbf{Neda Kalantari}}; Thesis: Studying possibility pf spectro-polarimetry observations from microlensing events.}
    \cvitem{2020}{\textit{\textbf{Fatemeh Kazemian}}; Thesis: Mass-Velocity Dispersion Relation by using the Gaia Data and its effect on short-duration microlensing events (\textbf{Results published}).}
   \cvitem{2019}{\textit{\textbf{Parisa Sangtarash}}; Thesis: Gravitational microlensing from limb-darkened source stars (\textbf{Results published}).}	
   	\cvitem{2018}{\textit{\textbf{Banafshe Adami}}; Thesis: A review on data-reduction process of microlensing events (\textbf{Results published}).}
	\cvitem{2017}{\textit{\textbf{Mariyam Javadizadeh}}; Thesis: Circumstellar disk perturbations on microlensing light curves.}

	\subsection{\textbf{PhD. projects}}
	\cvitem{2021-}{\textit{\textbf{Parisa Sangtarash}}; Thesis: Gravitational Microlensing of variable stars toward M31.}
	\cvitem{2020}{\textit{\textbf{Elahe Khalouei}}; Co-supervisor; Thesis: Measuring stellar atmosphere parameters using follow-up polarimetric microlensing observations (\textbf{Results published}).}
	\cvitem{2019}{\textit{\textbf{Fatemeh Bagheri}}; Advisor; Thesis: Exoplanet detection through space-based microlensing observation (\textbf{Results published}).}
\end{section}


\begin{section}{Publications}
\nocite{*}
\leavevmode\printbibliography[heading=none]
%\printbibliography[sorting=nyt]
\end{section}


%%%%%%%%%%%%%%%%%%%%%%%%%%%%%%%%%%%%%%%%%
\begin{section}{Selected Talks in Conferences \& Meetings} 
\cvitem{2022, Sep}{\textcolor{blue2}{\textbf{25th Microlensing Conference}}, Institut d'Astrophysique de Paris (IAP), Paris. Title: Sensitivity to habitable planets in the Roman microlensing survey.}
\cvitem{2021, Feb}{In weekly meeting, \textcolor{blue2}{\textbf{Physics Department, Isfahan university of Technology}}. Title: Detecting isolated blackholes in the Galactic disk through astrometric microlensing.}	
\cvitem{2021, April}{In weekly meeting, \textcolor{blue2}{\textbf{Physics Department, Cambridge university}}. Title: Gravitational microlensing: Observation and Interpretation.}	
\cvitem{2020, July}{In weekly meeting, \textcolor{blue2}{\textbf{Physics Department, Sharif University of Technology}}. Title: Microlensing and variable stars.}	
\cvitem{2018, Nov}{In monthly meeting, \textcolor{blue2}{\textbf{Isfahan physics club}}, Isfahan, Iran. Title: Observation with Danish 1.54m telescope.}	
\cvitem{2017, Dec}{In monthly meeting, \textcolor{blue2}{\textbf{Adib center of astronomy learning}}, Isfahan, Iran. Title: Planetary systems observations.}	
\cvitem{2016, March}{\textcolor{blue2}{\textbf{Annual meeting of the MiNDSTEp consortium}}, Salerno University, Italy. Title: The advantages of using Lucky Imaging camera for observations of planetary microlensing. }
\cvitem{2013, Jan}{\textcolor{blue2}{\textbf{National Conference on Gravitation and Cosmology}}, Shahid Beheshti Uiversity, Iran. Title: Astrometric properties of gravitational lenses.}
\cvitem{2012, May}{\textcolor{blue2}{\textbf{IPM Physics Spring Conference}}, IPM, Iran. Title: Detecting gravitational microlensing by Lucky imaging camera.}		
\cvitem{2010, May}{\textcolor{blue2}{\textbf{Meeting on Research in Astronomy}}, IASBS, Zanjan. Title: Detecting hot Jupiters in caustic crossing.}		
\end{section} 

%%%%%%%%%%%%%%%%%%%%%%%%%%%%%%%%%%%%%%%%%
\begin{section}{Computer skills} 
\cvitem{Programming Languages}{Python, C, C++, IDL, Fortran, Mathematica}
\cvitem{Scientific Tools}{Iraf (worked with DAOPHOT), DS9, DANDIA}
\cvitem{Database}{SQL, pandas, Machine Learning (worked with scikit-learn)}
\end{section} 

%%%%%%%%%%%%%%%%%%%%%%%%%%%%%%%%%%%%%%%%%
\begin{section}{Interests}
\cvitem{Sport}{Mountain climbing, Running}	
\cvitem{Reading book. }{I enjoy reading books very much, especially self-help books, novels, etc. }	
%\cvitem{}{\textit{\textbf{}} .}	
\end{section}  


\section{Referees}
\begin{tabular}{lr}
	\begin{minipage}[t]{3in}
		\textbf{Prof. Uffe G. {J{\o}rgensen}}\\
		Professor, Niels Bohr Institute\\
		Denmark\\
	    (+45) 61306640\\
		\Letter\ \href{}{\textcolor{blue1}{uffegj@nbi.ku.dk}}
	\end{minipage}
	\\
	\\
	\begin{minipage}[t]{3in}
		\textbf{Prof.  Richard Ignace}\\
		Professor, Department of Physics \\
		East Tennessee state university\\
		(+1) 4234396904\\
		\Letter\ \href{}{\textcolor{blue1}{IGNACE@mail.etsu.edu}}
	\end{minipage}
    \\
    \\
	\begin{minipage}[t]{3in}
		\textbf{Dr. Markus Hundertmark}\\
		Researcher, Department of Physics \\
		Heidelberg University\\
		(+49) 6221541867\\
		\Letter\ \href{}{\textcolor{blue1}{markus.hundertmark@uni-heidelberg.de}}
	\end{minipage}
	\\
\end{tabular}




\end{document}